\documentclass[12pt]{amsart}
\usepackage{url,amsmath,amsthm,enumitem,amsfonts,tikz,verbatim,amssymb, wasysym, clrscode3e, booktabs}
\usepackage[makeroom]{cancel}
\usetikzlibrary{arrows}
\date{19 February 2020}

\title{DEPARTMENT OF MATHEMATICAL SCIENCES \protect\\
MATH 3570 \protect\\
Senior Seminar I}
\author{Joel Savitz \\ Spring 2020 \\ SID: 01739537}

\begin{document}

\maketitle

\section*{A Mathematical Aspect to Raspberry Pi Development}

From summer 2019 through the present, a group of computer science students and I have worked on a project in collaboration with Red Hat --- the software company --- to improve the usability of Fedora Linux on the Raspberry Pi hardware platform in particular and the 64-bit arm architecture in general.

Our aim is to enable Jeff Brown and Professor Bill Moloney to use Fedora Linux as a basis for the curriculum of their popular IoT class that runs each spring.
We began by identifying existing usability flaws, such as missing or misconfigured features, improperly integrated hardware, and performance bottlenecks. Once we had established this baseline of knowledge, we chose a few individual projects to test the waters over the Fall 2019 semester. Temporarily shelving more rigorous performance analysis, we worked on a GPIO syntax compatibility library, a usable set of starting software for students, and operating system images that could be readily copied to microSD cards for student use.



\pagebreak
\noindent\begin{tabular}{ll}
	\makebox[4in]{\hrulefill} & \makebox[1.5in]{\hrulefill} \\
	Joel Savitz, Student & Date \\[8ex]
	\makebox[4in]{\hrulefill} & \makebox[1.5in]{\hrulefill} \\
	Dr. Kenneth Levasseur, Advisor & Date \\[8ex]
	\makebox[4in]{\hrulefill} & \makebox[1.5in]{\hrulefill} \\
	Dr. Aida Kadic-Galeb, Coordinator & Date \\[8ex]
\end{tabular}



\end{document}
